% Hlavicka pro protokoly z fyzikalniho praktika.
% Verze pro: LaTeX
% Verze hlavicky: 22. 2. 2007
% Autor: Ustav fyziky kondenzovanych latek
% Ke stazeni: www.physics.muni.cz/ufkl/Vyuka/
% Licence: volne k pouziti, nejlepe k vcasnemu odevzdani protokolu z Vaseho mereni.


\documentclass[czech,11pt,a4paper]{article}
\usepackage[T1]{fontenc}
\usepackage{graphicx}
\usepackage{mathtools}
\usepackage{amssymb}
\usepackage{amsthm}
\usepackage{thmtools}
\usepackage{xcolor}
\usepackage{nameref}
\usepackage{babel}
\usepackage{hyperref}


%%% Nemente:
\usepackage[margin=2cm]{geometry}
\newtoks\jmenopraktika \newtoks\jmeno \newtoks\datum
\newtoks\obor \newtoks\skupina \newtoks\rocnik \newtoks\semestr
\newtoks\cisloulohy \newtoks\jmenoulohy
\newtoks\tlak \newtoks\teplota \newtoks\vlhkost
%%% Nemente - konec.


%%%%%%%%%%% Doplnte pozadovane polozky:

\jmenopraktika={Fyzikální praktikum 1}  % nahradte jmenem vaseho predmetu
\jmeno={Teodor Duraković}            % nahradte jmenem mericiho
\datum={28.~února 2024}        % nahradte datem mereni ulohy
\obor={F}                     % nahradte zkratkou vami studovaneho oboru
\skupina={St 8:00}            % nahradte dobou vyuky vasi seminarni skupiny
\rocnik={I}                  % nahradte rocnikem, ve kterem studujete
\semestr={II}                 % nahradte semestrem, ve kterem studujete

\cisloulohy={1}               % nahradte cislem merene ulohy
\jmenoulohy={Měření hustoty válce} % nahradte jmenem merene ulohy

\tlak={98 \,\rm 900}                   % nahradte tlakem pri mereni (v hPa)
\teplota={20.2}               % nahradte teplotou pri mereni (ve stupnich Celsia)
\vlhkost={43.5}               % nahradte vlhkosti vzduchu pri mereni (v %)

%%%%%%%%%%% Konec pozadovanych polozek.


%%%%%%%%%%% Uzitecne balicky:

%%%%%% Zamezeni parchantu:
\widowpenalty 10000 \clubpenalty 10000 \displaywidowpenalty 10000
%%%%%% Parametry pro moznost vsazeni vetsiho poctu obrazku na stranku
\setcounter{topnumber}{3}	  % max. pocet floatu nahore (specifikace t)
\setcounter{bottomnumber}{3}	  % max. pocet floatu dole (specifikace b)
\setcounter{totalnumber}{6}	  % max. pocet floatu na strance celkem
\renewcommand\topfraction{0.9}	  % max podil stranky pro floaty nahore
\renewcommand\bottomfraction{0.9} % max podil stranky pro floaty dole
\renewcommand\textfraction{0.1}	  % min podil stranky, ktery musi obsahovat text
\intextsep=8mm \textfloatsep=8mm  %\intextsep pro ulozeni [h] floatu a \textfloatsep pro [b] or [t]

% Tecky za cisly sekci:
\renewcommand{\thesection}{\arabic{section}.}
\renewcommand{\thesubsection}{\thesection\arabic{subsection}.}
% Jednopismenna mezera mezi cislem a nazvem kapitoly:
\makeatletter \def\@seccntformat#1{\csname the#1\endcsname\hspace{1ex}} \makeatother


%%%%%%%%%%%%%%%%%%%%%%%%%%%%%%%%%%%%%%%%%%%%%%%%%%%%%%%%%%%%%%%%%%%%%%%%%%%%%%%
%%%%%%%%%%%%%%%%%%%%%%%%%%%%%%%%%%%%%%%%%%%%%%%%%%%%%%%%%%%%%%%%%%%%%%%%%%%%%%%
% Zacatek dokumentu
%%%%%%%%%%%%%%%%%%%%%%%%%%%%%%%%%%%%%%%%%%%%%%%%%%%%%%%%%%%%%%%%%%%%%%%%%%%%%%%
%%%%%%%%%%%%%%%%%%%%%%%%%%%%%%%%%%%%%%%%%%%%%%%%%%%%%%%%%%%%%%%%%%%%%%%%%%%%%%%

\begin{document}
	
	%%%%%%%%%%%%%%%%%%%%%%%%%%%%%%%%%%%%%%%%%%%%%%%%%%%%%%%%%%%%%%%%%%%%%%%%%%%%%%%
	% Nemente:
	%%%%%%%%%%%%%%%%%%%%%%%%%%%%%%%%%%%%%%%%%%%%%%%%%%%%%%%%%%%%%%%%%%%%%%%%%%%%%%%
	\thispagestyle{empty}
	
	{
		\begin{center}
			\sf 
			{\Large Ústav fyzikální elektroniky Přírodovědecké fakulty Masarykovy univerzity} \\
			\bigskip
			{\huge \bfseries FYZIKÁLNÍ PRAKTIKUM} \\
			\bigskip
			{\Large \the\jmenopraktika}
		\end{center}
		
		\bigskip
		
		\sf
		\noindent
		\setlength{\arrayrulewidth}{1pt}
		\begin{tabular*}{\textwidth}{@{\extracolsep{\fill}} l l}
			\large {\bfseries Zpracoval:}  \the\jmeno & \large  {\bfseries Naměřeno:} \the\datum\\[2mm]
			\large  {\bfseries Obor:} \the\obor  \hspace{40mm}  {\bfseries Skupina:} \the\skupina %
			%{\bfseries Ročník:} \the\rocnik \hspace{5mm} {\bfseries Semestr:} \the\semestr  
			&\large {\bfseries Testováno:}\\
			\\
			\hline
		\end{tabular*}
	}
	
	\bigskip
	
	{
		\sf
		\noindent \begin{tabular}{p{3cm} p{0.6\textwidth}}
			\Large  Úloha č. {\bfseries \the\cisloulohy:} \par
			\smallskip
			$T=\the\teplota$~$^\circ$C \par
			$p=\the\tlak$~Pa \par
			$\varphi=\the\vlhkost$~\%
			&\Large \bfseries \the\jmenoulohy  \\[2mm]
		\end{tabular}
	}
	
	\vskip1cm
	
	%%%%%%%%%%%%%%%%%%%%%%%%%%%%%%%%%%%%%%%%%%%%%%%%%%%%%%%%%%%%%%%%%%%%%%%%%%%%%%%
	% konec Nemente.
	%%%%%%%%%%%%%%%%%%%%%%%%%%%%%%%%%%%%%%%%%%%%%%%%%%%%%%%%%%%%%%%%%%%%%%%%%%%%%%%
	
	%%%%%%%%%%%%%%%%%%%%%%%%%%%%%%%%%%%%%%%%%%%%%%%%%%%%%%%%%%%%%%%%%%%%%%%%%%%%%%%
	%%%%%%%%%%%%%%%%%%%%%%%%%%%%%%%%%%%%%%%%%%%%%%%%%%%%%%%%%%%%%%%%%%%%%%%%%%%%%%%
	% Zacatek textu vlastniho protokolu
	%%%%%%%%%%%%%%%%%%%%%%%%%%%%%%%%%%%%%%%%%%%%%%%%%%%%%%%%%%%%%%%%%%%%%%%%%%%%%%%
	%%%%%%%%%%%%%%%%%%%%%%%%%%%%%%%%%%%%%%%%%%%%%%%%%%%%%%%%%%%%%%%%%%%%%%%%%%%%%%%
	
	
	\section{Zadání}
	Zjistit hustotu dutého válce pomocí změření jeho rozměrů a hmotnosti.
	
	\section{Postup}
	Válci jsou změřeny veškeré dimenze (vnější a vnitřní diametr, výška). Průměry jsou měřeny posuvným měřítkem ($d = 0.02\,\rm mm$), výška mikrometrem ($d = 0,005\,\rm mm$). Hmotnost je zvážena laboratorními váhami ($ d = 0,0001\,\rm g ; e = 0,001\,\rm g$). Na závěr je těleso ponořeno do vody, čímž je odhadnuta jeho hustota - těleso neplove, hustota tudíž bude vyšší než $1000\,\rm {kg}.{m^{-3}}$.
	
	
	\subsection{Měření}
	Měření rozměrů jsou vždy provedena desetkrát, při zpozorování viditelných hrubých chyb je měření opakováno bez zápisu chybné hodnoty. 
	\\
	
	\quad \quad \quad\quad\quad \quad\quad\quad\quad\quad\quad\quad\quad \begin{tabular}{||r|c|c|c||}
		\hline
		n & D [cm] & d [cm] & h [cm] \\
		\hline
		1 & 4.962 & 0.992 & 1.5040 \\
		\hline
		2 & 4.966 & 1.000 &1.5115\\
		\hline
		3 & 4.962 &0.980 &1.5110 \\
		\hline
		4 & 4.962 & 0.972 &1.5135\\
		\hline
		5 &  4.964 &1.000 &1.5120\\
		\hline
		6 &  4.962 & 0.996&1.5065\\
		\hline 
		7 &  4.964 & 0.994&1.5105\\
		\hline
		8 &  4.962 &1.000 &1.5085\\
		\hline
		9 &  4.964 & 1.000&1.5080\\
		\hline
		10 &  4.962 & 0.988&1.5125\\
		\hline
	\end{tabular} \\
	

\quad \quad \quad\quad\quad \quad\quad\quad\quad\quad\quad\quad Změřená hmotnost činila $m = 33,2557\,\rm g$
	
	\subsection{Zpracování měření}
	Vztahem \begin{equation}
		\overline{x} = \frac{1}{N} \sum_{i=1}^{N} x_i
	\end{equation}
	získáme odhady středních hodnot (arit. průměry) vícekrát měřených veličin (uvedeny v bodě 2.6.). \\
	Střední hodnoty dosadíme do formule pro výpočet hustoty:
	\begin{equation}
		 \rho = \frac m V = \frac{ m}{\pi  h \left( (\frac { D} 2)^2 - (\frac {{d}} 2)^2)\right)} = 1.18599 \,\rm g.cm^{-3} = 1 185.99 \,\rm kg.m^{-3}
	\end{equation}
	Vztahem
	\begin{equation}
		\sigma = \sqrt[]{\frac{\sum_{i =1}^N{ (x_i - \overline{x} )^2} }{N-1}}
	\end{equation}
	získáme odhad směrodatné odchylky. Úpravou Studentovým koeficientem s $p = 0,9973, \nu = 9$ získáme hrubé chyby (krajní odchylky) pro měřené veličiny.
    Vidíme, že měřené hodnoty z intervalů nevystupují, soubory hodnot tudíž není třeba nijak upravovat.	
	
	\subsection{Nejistoty typu A}
	Nejistoty typu A získáme užitím vztahu
	\begin{equation}
		u_x = \sqrt[]{\frac{\sum_{i =1}^N{ (x_i - \overline{x} )^2} }{N(N-1)}}
	\end{equation}
	
	\subsection{Nejistoty typu B}
	Nejistoty typu B získáme užitím vztahu $ u_B = a/k $
	
	\subsubsection{Měřidla délky}
	pro měřidla délky platí: $a = d; k = \sqrt{3}; u_b = \frac{d}{\sqrt{3}}$ \\
	Pro posuvné měřítko tedy nejistota typu B činí $u_B = \frac {0.002}{ \sqrt{3}} = 0.00115 cm$ \\
	Pro mikrometr platí $u_B = \frac{0.0005}{\sqrt{3}} = 0.000289 \,\rm cm$ \\
\textit{	Zde je třeba konstatovat, že pro $a$ posuvného měřítka je používána celá hodnota nejmenšího dílku, zatímco u mikrometru je uvedena jeho polovina. Je to provedeno proto, že jsem při měření byl u mikrometru schopen určit polovinu dílku, zatímco u posuvného měřítka nikoliv.}
	
	\subsubsection{Váhy}
	Pro váhy platí: $a=e; k =3$, tedy $ u_B = \frac{0.001}{3} = 0.000333\,\rm g$
	
	\subsection{Nejistota typu C}
	Nejistotu typu C získáme vztahem:
	
	\begin{equation}
		u_C = \sqrt{u_A ^2 + u_B ^2}
	\end{equation}
	
	\subsection{Spočítané veličiny}
	Výše uvedenými vztahy jsme získali následující veličiny \\
	
	\quad \quad \quad \quad \quad \quad \quad   \begin{tabular}{||l|c|c|c|c|c||}
		\hline
		& $\overline{x}$ & $\hat{k}$ & $u_A$ & $u_B$ & $u_C$   \\
		\hline
		D [cm]& 4.963 & 0.0058 & 0.000447 & 0.00115 & 0.00124   \\
		\hline
		d [cm]& 0.9922 & 0.039 & 0.00305 & 0.00115 & 0.00326   \\
		\hline
		h [cm]& 1.5098 & 0.012 & 0.000943 & 0.000289 & 0.000986   \\
		\hline
		m [g]& 33.2557 & - & - & 0.000333 & 0.000333   \\
		\hline
	\end{tabular}
	
	
	\subsection{Zákon přenosu nejistot}
	
	
	Užitím zákona přenosu nejistot získáme formuli pro nejistotu výsledku - hustoty:
	{\tiny
	\begin{equation}
		u_{\rho} = \sqrt[]{
			\left(\frac{-8 \pi h m D}{\left(\pi h\left(D^2-d^2\right)\right)^2}\right)^2 \cdot u_D^2 +
			\left(\frac{8 \pi h m d}{\left(\pi h\left(D^2-d^2\right)\right)^2}\right)^2 \cdot u_d^2  +
			\left(\frac{-4 m}{\pi h^2 \cdot\left(D^2-d^2\right)}\right)^2 
			\cdot u_h^2 +
			\left(\frac{4 }{\pi h \cdot\left(D^2-d^2\right)}\right)^2 \cdot u_m^2 }
 	\end{equation}
}
	Po dosazení získáváme kombinovanou nejistotu hustoty:
	\begin{equation*}
		u_\rho = 0.0034 \,\rm g .cm^{-3}
	\end{equation*}
	Tuto nejistotu upravíme studentovým koeficientem pro $ p = 0.68 ; \nu = 9$ :
	
	\begin{equation*}
		U_{ \rho } = 1.059 * u_{\rho} =   0.0036 \,\rm g .cm^{-3}=  3.6 \,\rm kg.m^{-3}
	\end{equation*}
	
	\section{Výsledek}
	Výše popsaným postupem jsme získali hodnotu hustoty:
	
	\begin{equation*}
		\rho = (1186 \pm 4) \,\rm kg.m^{-3} (p = 0.6827)  
	\end{equation*}
	
	\section{Závěr}
	Spočítaná hodnota se pohybuje v předpokládaných mezích, bez znalosti konkrétního druhu materiálu však hodnotu nedokážeme porovnat s hodnotou tabulkovou. Přesnost experimentu lze nicméně částečně zhodnotit relativní odchylkou, která je při hodnotě $ r_{\rho} = \frac {4}{1186} = 0.0034$ - tedy kolem tří promile - s~ohledem na účely experimentu přijatelná.  
	
	% Nakonec nezapomeňte projet text programem vlna nebo vlnka, např.
	% 	vlna -m -l -n mojeuloha.tex
	% nebo zkontrolovat a opravit jednopísmenné předložky na koncích řádků ručně.
	
	
\end{document}