% Hlavicka pro protokoly z fyzikalniho praktika.
% Verze pro: LaTeX
% Verze hlavicky: 22. 2. 2007
% Autor: Ustav fyziky kondenzovanych latek
% Ke stazeni: www.physics.muni.cz/ufkl/Vyuka/
% Licence: volne k pouziti, nejlepe k vcasnemu odevzdani protokolu z Vaseho mereni.


\documentclass[czech,11pt,a4paper]{article}
\usepackage[T1]{fontenc}
\usepackage{graphicx}
\usepackage{mathtools}
\usepackage{amssymb}
\usepackage{amsthm}
\usepackage{thmtools}
\usepackage{xcolor}
\usepackage{nameref}
\usepackage{babel}
\usepackage{hyperref}
\usepackage{multicol}
\usepackage[export]{adjustbox}
\usepackage{subcaption}
\usepackage{caption}
\usepackage{multirow}
\usepackage{float}
\usepackage{placeins}




%%% Nemente:
\usepackage[margin=2cm]{geometry}
\newtoks\jmenopraktika \newtoks\jmeno \newtoks\datum
\newtoks\obor \newtoks\skupina \newtoks\rocnik \newtoks\semestr
\newtoks\cisloulohy \newtoks\jmenoulohy
\newtoks\tlak \newtoks\teplota \newtoks\vlhkost
%%% Nemente - konec.


%%%%%%%%%%% Doplnte pozadovane polozky:

\jmenopraktika={Fyzikální praktikum 1}  % nahradte jmenem vaseho predmetu
\jmeno={Teodor Duraković}            % nahradte jmenem mericiho
\datum={24.~dubna 2024}        % nahradte datem mereni ulohy
\obor={F}                     % nahradte zkratkou vami studovaneho oboru
\skupina={St 8:00}            % nahradte dobou vyuky vasi seminarni skupiny
\rocnik={I}                  % nahradte rocnikem, ve kterem studujete
\semestr={II}                 % nahradte semestrem, ve kterem studujete

\cisloulohy={3}               % nahradte cislem merene ulohy
\jmenoulohy={Měření viskozity, hustoty a povrchového napětí kapalin} % nahradte jmenem merene ulohy

\tlak={97 \,\rm 814}                   % nahradte tlakem pri mereni (v hPa)
\teplota={20.3}               % nahradte teplotou pri mereni (ve stupnich Celsia)
\vlhkost={33.5}               % nahradte vlhkosti vzduchu pri mereni (v %)

%%%%%%%%%%% Konec pozadovanych polozek.


%%%%%%%%%%% Uzitecne balicky:

%%%%%% Zamezeni parchantu:
\widowpenalty 10000 \clubpenalty 10000 \displaywidowpenalty 10000
%%%%%% Parametry pro moznost vsazeni vetsiho poctu obrazku na stranku
\setcounter{topnumber}{3}	  % max. pocet floatu nahore (specifikace t)
\setcounter{bottomnumber}{3}	  % max. pocet floatu dole (specifikace b)
\setcounter{totalnumber}{6}	  % max. pocet floatu na strance celkem
\renewcommand\topfraction{0.9}	  % max podil stranky pro floaty nahore
\renewcommand\bottomfraction{0.9} % max podil stranky pro floaty dole
\renewcommand\textfraction{0.1}	  % min podil stranky, ktery musi obsahovat text
\intextsep=8mm \textfloatsep=8mm  %\intextsep pro ulozeni [h] floatu a \textfloatsep pro [b] or [t]

% Tecky za cisly sekci:
\renewcommand{\thesection}{\arabic{section}.}
\renewcommand{\thesubsection}{\thesection\arabic{subsection}.}
\renewcommand{\thesubsubsection}{\thesubsection\arabic{subsubsection}.}
% Jednopismenna mezera mezi cislem a nazvem kapitoly:
\makeatletter \def\@seccntformat#1{\csname the#1\endcsname\hspace{1ex}} \makeatother


%%%%%%%%%%%%%%%%%%%%%%%%%%%%%%%%%%%%%%%%%%%%%%%%%%%%%%%%%%%%%%%%%%%%%%%%%%%%%%%
%%%%%%%%%%%%%%%%%%%%%%%%%%%%%%%%%%%%%%%%%%%%%%%%%%%%%%%%%%%%%%%%%%%%%%%%%%%%%%%
% Zacatek dokumentu
%%%%%%%%%%%%%%%%%%%%%%%%%%%%%%%%%%%%%%%%%%%%%%%%%%%%%%%%%%%%%%%%%%%%%%%%%%%%%%%
%%%%%%%%%%%%%%%%%%%%%%%%%%%%%%%%%%%%%%%%%%%%%%%%%%%%%%%%%%%%%%%%%%%%%%%%%%%%%%%

\begin{document}
	
	%%%%%%%%%%%%%%%%%%%%%%%%%%%%%%%%%%%%%%%%%%%%%%%%%%%%%%%%%%%%%%%%%%%%%%%%%%%%%%%
	% Nemente:
	%%%%%%%%%%%%%%%%%%%%%%%%%%%%%%%%%%%%%%%%%%%%%%%%%%%%%%%%%%%%%%%%%%%%%%%%%%%%%%%
	\thispagestyle{empty}
	
	{
		\begin{center}
			\sf 
			{\Large Ústav fyziky a technologií plazmatu Přírodovědecké fakulty Masarykovy univerzity} \\
			\bigskip
			{\huge \bfseries FYZIKÁLNÍ PRAKTIKUM} \\
			\bigskip
			{\Large \the\jmenopraktika}
		\end{center}
		
		\bigskip
		
		\sf
		\noindent
		\setlength{\arrayrulewidth}{1pt}
		\begin{tabular*}{\textwidth}{@{\extracolsep{\fill}} l l}
			\large {\bfseries Zpracoval:}  \the\jmeno & \large  {\bfseries Naměřeno:} \the\datum\\[2mm]
			\large  {\bfseries Obor:} \the\obor  \hspace{40mm}  {\bfseries Skupina:} \the\skupina %
			%{\bfseries Ročník:} \the\rocnik \hspace{5mm} {\bfseries Semestr:} \the\semestr  
			&\large {\bfseries Testováno:}\\
			\\
			\hline
		\end{tabular*}
	}
	
	\bigskip
	
	{
		\sf
		\noindent \begin{tabular}{p{3cm} p{0.6\textwidth}}
			\Large  Úloha č. {\bfseries \the\cisloulohy:} \par
			\smallskip
			$T=\the\teplota$~$^\circ$C \par
			$p=\the\tlak$~Pa \par
			$\varphi=\the\vlhkost$~\%
			&\Large \bfseries \the\jmenoulohy  \\[2mm]
		\end{tabular}
	}
	
	\vskip1cm
	
	%%%%%%%%%%%%%%%%%%%%%%%%%%%%%%%%%%%%%%%%%%%%%%%%%%%%%%%%%%%%%%%%%%%%%%%%%%%%%%%
	% konec Nemente.
	%%%%%%%%%%%%%%%%%%%%%%%%%%%%%%%%%%%%%%%%%%%%%%%%%%%%%%%%%%%%%%%%%%%%%%%%%%%%%%%
	
	%%%%%%%%%%%%%%%%%%%%%%%%%%%%%%%%%%%%%%%%%%%%%%%%%%%%%%%%%%%%%%%%%%%%%%%%%%%%%%%
	%%%%%%%%%%%%%%%%%%%%%%%%%%%%%%%%%%%%%%%%%%%%%%%%%%%%%%%%%%%%%%%%%%%%%%%%%%%%%%%
	% Zacatek textu vlastniho protokolu
	%%%%%%%%%%%%%%%%%%%%%%%%%%%%%%%%%%%%%%%%%%%%%%%%%%%%%%%%%%%%%%%%%%%%%%%%%%%%%%%
	%%%%%%%%%%%%%%%%%%%%%%%%%%%%%%%%%%%%%%%%%%%%%%%%%%%%%%%%%%%%%%%%%%%%%%%%%%%%%%%
	
	
	\section{Zadání}
	Určit \\
	1. teplotní závislost viskozity vody Ubbelohdeho viskozimetrem\\
	2. viskozitu vody metodou výtoku z Mariottovy láhve\\
	3. hustotu lihu metodou a) pyknometrickou a b) ponorného tělíska\\
	4. povrchové napětí destilované vody a lihu metodou du Noüyho metodou kroužku\\
	5. kontaktní úhly vody a methylen jodidu.
	
	\section{Postup, metody měření}
	\subsection{Viskozita vody}
	\subsubsection{Metoda Ubbelohdeho viskozimetru}
	Ubbelhodeho viskozimetr se používá ke stanovení kinematické viskozity kapalin. Platí vztah:
	\begin{equation}
		\nu = Kt
	\end{equation}
	kde $\nu$ je kinematická viskozita, $K$ časová konstanta viskosimetru a $t$ čas změny hladiny kapaliny \\v trubici mezi dvěma vyznačenými úrovněmi. Měření provádíme třikrát, při teplotě kapaliny cca $20, 30$ a $40 ^\circ C$.
	\subsubsection{Metoda Mariottovy láhve}
	Mariottova láhev zajišťuje konstantní tlakový spád mezi konci trubice. Proudění lze při výtoku považovat za laminární a proto platí vztah:
	\begin{equation}
		\eta = \frac{\pi R^4 pt}{8VL} = \frac{\pi \rho R^4 \, \rho g h \,t}{8mL}
	\end{equation}
	kde $R$ je poloměr trubice, $p$ je rozdíl tlaků mezi konci trubice, $t$ čas, za který vyteče z nádoby objem  $V$, $L$ je délka trubice a $h$ výškový rozdíl mezi trubicí a kapilárou lahve.
	\subsection{Hustota lihu}
	\subsubsection{Pyknometrická metoda}
	Metoda pyknometru je založena na tom, že nádoba u referenční i zkoumané kapaliny pojme identický objem. Platí:
	\begin{equation}
		\rho = (\rho_k - \rho_v) \frac{ m - m_p}{m_k - m_p} + \rho_v
	\end{equation}
	kde $\rho_k$ je hustota kalibrační kapaliny (v našem případě destilované vody), $\rho_v$ hustota vzduchu,\\ $m_p$ hmotnost pyknometru, $m_k$ hmotnost kalibrační kapaliny a $m$ hmotnost lihu. 
	\subsubsection{Metoda ponorného tělíska}
	Na váhy zavěsíme závaží, váhy vytárujeme a kapalinu ve válci postavíme pod váhy. Při plně ponořeném závaží zaznamenáme měřenou hmotnost při obou kapalinách (měřené a kalibrační). 
	Zde aplikujeme formuli
	\begin{equation}
		\rho = \frac{m}{m_k} \rho_i
	\end{equation}
	\subsection{Povrchové napětí}
	\subsubsection{Metoda kroužku}
	V souladu s návodem měříme sílu při odtržení kroužku. Používáme vztah
	\begin{equation}
		\sigma = \frac{F_{max}}{4\pi R} \cdot f
	\end{equation}
	kde $F$ je maximální síla při odtržení kroužku, $2\pi R$ je obvod kroužku a $f$ je Harkins-Jordanův koeficient, jehož hodnota činí přibližně $f = 0.77$.
	\subsubsection{metoda kontaktního úhlu}
	Měríme dispersní složku povrchové energie vody a kalibrační kapaliny. Používáme formuli
	\begin{equation}
		\frac{\sigma_{\mathrm{H}_2 \mathrm{O}}^{l w}}{\sigma_{\mathrm{H}_2 \mathrm{O}}}=\frac{\sigma_{k a l}^{l w}}{\sigma_{k a l}} \frac{\sigma_{\mathrm{H}_2 \mathrm{O}}}{\sigma_{k a l}}\left(\frac{1+\cos \theta_{\mathrm{H}_2 \mathrm{O}}}{1+\cos \theta_{k a l}}\right)^2
	\end{equation}
	kde $\sigma_k$ jsou jmenovité hodnoty povrchové energie vody a kalibrační kapaliny, $\sigma ^{lw}_k$ jsou Lifshitz - van der Waalsovy složky povrchového napětí obsahující coulombickou, indukční a dispersní složku povrchové energie. $\Theta_k $ značí kontaktní úhly na kapkách přichycených na teflonovém povrchu$^{[3]}$.
	\section{Měření}
	\subsection{Viskozita vody}
	\subsubsection{Ubbelohdeho viskozimetr}
	Pro kalkulaci používáme vztah (1), časová konstanta pro použitý viskozimetr je rovna \\ $K = 1.063 \cdot 10^{-3 } \,\rm mm^2 \,s^{-2},$ $ r(K) = 0.65 \% \, (p \,95\%)$ 
	
   Získáváme:
   	
   	\begin{center}
   		
   	\begin{tabular}{c|c|c}
   		\hline
   		T [°C] & t [s] & $\eta \,\rm [mm^2 s^-1]$ \\
   		21.0   & 947.9 & $ 1.008 \pm 0.007$ \\
   		28.8   & 776.4 & $0.825 \pm 0.005$  \\
   		38.9   & 628.3 & $0.668 \pm 0.004$ 
   	\end{tabular}
   	
   	\end{center}
   	
   	\subsubsection{Mariottova láhev}
   	Měřením získáváme následující hodnoty:
   	\begin{center}
   		\begin{tabular}{l|l}
   		\hline
   		
   		R      & $0.570 \pm 0.001 \,\rm mm$  \\
   		m      & $74.190 \pm 0.003 \,\rm g$  \\
   		L      & $165.0 \pm 0.5 \,\rm mm$    \\
   		t      & $319.86 \pm 0.3 \,\rm s$    \\
   		h      & $112.00 \pm 0.05 \,\rm mm$  \\
   		$\rho^{[1]}$  & $0.9982 \,\rm g.cm^{-3}$        
   	\end{tabular}
   	
   	\end{center}
   Po vložení hodnot do formule (2) získáváme hodnotu $\eta = 1.186 \pm 0.009 \,\rm mPa.s$.
   \subsection{Hustota lihu}
   \subsubsection{Pyknometrická metoda}
   Měříme hodnoty:
   \begin{center}
   	\begin{tabular}{l|l}
   		\hline
   		$m_p$    & $23.868 \pm 0.003 \,\rm g$ \\
   		$m_k$    & $74.042 \pm 0.003 \,\rm g$ \\
   		$m$      & $64.070 \pm 0.003 \,\rm g$ \\
   		$\rho_v ^{[2]}$ & $0.00116 \,\rm g.cm^{-3}$       \\
   		$\rho_k ^{[1]}$ & $0.9982 \,\rm g.cm^{-3}$     \\
   	\end{tabular}
   \end{center}
   Po aplikaci formule (3) získáváme: $\rho = 800.04 \pm 0.08 \,\rm kg . m^{-3}$.
   	
   \subsubsection{Metoda ponorného tělíska}
   Získáváme hodnoty:
   \begin{center}
   	\begin{tabular}{l|l}
   		\hline
   		$m$      & $3.828 \pm 0.003 \,\rm g$ \\
   		$m_k$    & $4.814 \pm 0.003 \,\rm g$ \\
   		$\rho_k^{[1]}$ & $0.9982 \,\rm g.cm^{-3}$   
   	\end{tabular}
   \end{center}
   Po dosazení do formule (4) získáváme $\rho = 793.7 \pm 0.8 \,\rm kg. m^{-3}$.
   \subsection{Povrchové napětí}
   \subsubsection{Metoda kroužku}
   Získáváme:
   
   \begin{center}
   	\begin{tabular}{c|c}
   		\hline
   		$F_{max} \,\rm $ - líh [mN] & $F_{max} \,\rm - voda$ [mN]\\ 
   		\hline
   		11.76               & 31.79                \\
   		11.50               & 32.15                \\
   		11.06               & 33.01                \\
   		11.68               & 34.73                \\
   		12.34               & 34.17                \\
   		11.96               & 33.86                \\
   		10.18               & 35.20                \\
   		10.54               & 34.75                \\
   		11.82               & 34.85                \\
   		11.62               & 34.60                \\
   		11.15               & 33.50               
   	\end{tabular}
   \end{center}
   Po dosazení do formule (5) spolu s poloměrem kroužku $R = 29 \pm 0.05\,\rm mm$ získáváme hodnoty:\\ $\sigma_{lih} = 24.1 \pm 0.4 \,\rm mN.m^{-1}$ a $\sigma_{voda} = 71.6 \pm 0.7 \,\rm mN.m^{-1}$
   
   
   \subsubsection{Metoda kontaktního úhlu}
   Při použití glycerolu jakožto kalibrační kapaliny získáváme
   
   \begin{center}
   	\begin{tabular}{l|l}
   	\hline
   	$\Theta_{H_2O}[^\circ]$ & $\Theta_{kal} [^\circ]$ \\
   	77.7                    & 61.3                    \\
   	72.6                    & 54.3                    \\
   	75.4                    & 57.7                    \\
   	70.8                    & 56.8                    \\
   	71.6                    & 69                     
   \end{tabular}
   \end{center}
   
   Po dosazení průměrných hodnot do formule (6) získáváme: $\sigma ^{lw}_{H_2O} = 32.02 \,\rm mN.m^{-1}$
   
	
	\section{Závěr}
	Pro viskozitu vody získáváme metodou Ubbelohdeho viskozimetru uspokojivé výsledky, u Mariottovy láhve se však výsledek výrazně odchyluje od očekáváného. Hustota lihu je při obou použitých metodách měření v očekávaných mezích při očekáváné přesnosti. Povrchové napětí vody i lihu je při metodě měření kroužkem blízké tabulkovým hodnotám, u metody kontaktního úhlu se však měřená hodnota od hodnoty tabulkové odchyluje, což připisujeme zejména nepřesnosti při měření úhlu a kontaminaci vzorků.
	
	
	\section{Použitý kód}
	{\tiny \begin{verbatim}
		from uncertainties import *
		import numpy as np
		
		K = ufloat(1.063 *10**(-3), 6.9095*10**(-6))
		
		# eta = K t
		
		# Define the array of ufloat values for t
		t_array = np.array([ufloat(947.9, 0.1), ufloat(776.4, 0.1), ufloat(628.3, 0.1)])
		
		# Calculate nu for every t
		nu_array = K * t_array
		
		# Print the nu values
		for nu in nu_array:
		print(nu)
		
		#Mariott
		
		h1 = ufloat(16.17, 0.006)
		h2 = ufloat(16.20, 0.006)
		h3 = ufloat(16.20, 0.006)
		
		d1 = ufloat(5.0, 0.006)
		d2 = ufloat(4.98, 0.006)
		d3 = ufloat(4.99, 0.006)
		#vše v kg, m
		R = ufloat(0.570e-3, 0.001e-3)
		h = (h1+h2+h3 - d1 -d2 -d3)/300
		m = ufloat(74.190e-3, 0.003e-3)
		L = ufloat(16.5e-2, 0.05e-2)
		t = ufloat(319.86, 0.3)
		rho = ufloat(0.9982e3, 0)	
		g = ufloat(9.81, 0)
		
		
		eta = ((np.pi * rho**2 *R**4 * g * h*t)/(8 * m * L))
		print(eta*1000)
		
		#hustota-pyknometr-metoda
		
		mp = ufloat(23.868, 0.003)
		mk = ufloat(74.042, 0.003)
		m = ufloat(64.070, 0.003)
		
		mk = mk - mp
		m = m - mp
		print (mk, m)
		rhok = ufloat(0.9982, 0)
		rhov = ufloat(0.001157, 0)
		
		rho = (rhok - rhov) * ((m )/(mk)) + rhov
		print(rho*1000)
		
		#hustota-ponorna metoda
		m = ufloat(3.828, 0.003)
		mk = ufloat(4.814, 0.003)
		rhok = ufloat(0.9982, 0)
		
		rho = m/mk * rhok
		print(rho*1000)
		
		#Viskosita - krouzek
		
		Flih = ufloat(11.42, 0.19)
		Fvoda = ufloat(33.87, 0.34)
		R = ufloat (58.0, 0.1)
		
		sigmalih = (Flih/(2*np.pi*R)) * 0.77
		print(sigmalih*1000)
		sigmavoda = Fvoda/(2*np.pi*R) * 0.77
		print(sigmavoda*1000)
	\end{verbatim}}

	
	\section{Zdroje}
	[1] Water density calculator [on-line] \\
	Dostupný z WWW: https://www.omnicalculator.com/physics/water-viscosity      \newline                  
	[2] Air density calculator [on-line] \\
	Dostupný z WWW: https://www.omnicalculator.com/physics/air-density
	[3] ŠTYKS, Martin, MĚRĚNÍ KONTAKTNÍHO ÚHLU A POVRCHOVÉ ENERGIE NA POVRCHU POVLAKŮ BETA-FOSFOREČNANU	VÁPENATÉHO
	dostupné on-line: [https://dspace.cvut.cz/bitstream/handle/10467/103553/F2-BP-2022-Styks-Martin-BP\_STYKS\_MARTIN.pdf]
	
	% Nakonec nezapomeňte projet text programem vlna nebo vlnka, např.
	% 	vlna -m -l -n mojeuloha.tex
	% nebo zkontrolovat a opravit jednopísmenné předložky na koncích řádků ručně.
	
\end{document}
