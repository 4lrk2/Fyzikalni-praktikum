% Hlavicka pro protokoly z fyzikalniho praktika.
% Verze pro: LaTeX
% Verze hlavicky: 22. 2. 2007
% Autor: Ustav fyziky kondenzovanych latek
% Ke stazeni: www.physics.muni.cz/ufkl/Vyuka/
% Licence: volne k pouziti, nejlepe k vcasnemu odevzdani protokolu z Vaseho mereni.


\documentclass[czech,11pt,a4paper]{article}
\usepackage[T1]{fontenc}
\usepackage{graphicx, animate}
\usepackage{mathtools}
\usepackage{amssymb}
\usepackage{amsthm}
\usepackage{thmtools}
\usepackage{xcolor}
\usepackage{nameref}
\usepackage{babel}
\usepackage{hyperref}
\usepackage{multicol}
\usepackage[export]{adjustbox}
\usepackage{subcaption}
\usepackage{caption}
\usepackage{multirow}
\usepackage{float}
\usepackage{placeins}

\graphicspath{ {./images/} }
\usepackage[backend=biber,style=numeric]{biblatex}       % [1], [2], ...




\addbibresource{ref.bib}


%%% Nemente:
\usepackage[margin=2cm]{geometry}
\newtoks\jmenopraktika \newtoks\jmeno \newtoks\datum
\newtoks\obor \newtoks\skupina \newtoks\rocnik \newtoks\semestr
\newtoks\cisloulohy \newtoks\jmenoulohy

%%% Nemente - konec.


%%%%%%%%%%% Doplnte pozadovane polozky:

\jmenopraktika={Fyzikální praktikum 3}  % nahradte jmenem vaseho predmetu
\jmeno={Teodor Duraković}            % nahradte jmenem mericiho
\datum={27.~března 2024}        % nahradte datem mereni ulohy
\obor={F}                     % nahradte zkratkou vami studovaneho oboru
\skupina={Út 14:00}            % nahradte dobou vyuky vasi seminarni skupiny
\rocnik={II}                  % nahradte rocnikem, ve kterem studujete
\semestr={IV}                 % nahradte semestrem, ve kterem studujete

\cisloulohy={7}               % nahradte cislem merene ulohy
\jmenoulohy={{\large Určení teploty elektrického oblouku a OH radikálu}}           % nahradte vlhkosti vzduchu pri mereni (v %)

%%%%%%%%%%% Konec pozadovanych polozek.


%%%%%%%%%%% Uzitecne balicky:

%%%%%% Zamezeni parchantu:
\widowpenalty 10000 \clubpenalty 10000 \displaywidowpenalty 10000
%%%%%% Parametry pro moznost vsazeni vetsiho poctu obrazku na stranku
\setcounter{topnumber}{3}	  % max. pocet floatu nahore (specifikace t)
\setcounter{bottomnumber}{3}	  % max. pocet floatu dole (specifikace b)
\setcounter{totalnumber}{6}	  % max. pocet floatu na strance celkem
\renewcommand\topfraction{0.9}	  % max podil stranky pro floaty nahore
\renewcommand\bottomfraction{0.9} % max podil stranky pro floaty dole
\renewcommand\textfraction{0.1}	  % min podil stranky, ktery musi obsahovat text
\intextsep=8mm \textfloatsep=8mm  %\intextsep pro ulozeni [h] floatu a \textfloatsep pro [b] or [t]

% Tecky za cisly sekci:
\renewcommand{\thesection}{\arabic{section}.}
\renewcommand{\thesubsection}{\thesection\arabic{subsection}.}
\renewcommand{\thesubsubsection}{\thesubsection\arabic{subsubsection}.}
% Jednopismenna mezera mezi cislem a nazvem kapitoly:
\makeatletter \def\@seccntformat#1{\csname the#1\endcsname\hspace{1ex}} \makeatother


%%%%%%%%%%%%%%%%%%%%%%%%%%%%%%%%%%%%%%%%%%%%%%%%%%%%%%%%%%%%%%%%%%%%%%%%%%%%%%%
%%%%%%%%%%%%%%%%%%%%%%%%%%%%%%%%%%%%%%%%%%%%%%%%%%%%%%%%%%%%%%%%%%%%%%%%%%%%%%%
% Zacatek dokumentu
%%%%%%%%%%%%%%%%%%%%%%%%%%%%%%%%%%%%%%%%%%%%%%%%%%%%%%%%%%%%%%%%%%%%%%%%%%%%%%%
%%%%%%%%%%%%%%%%%%%%%%%%%%%%%%%%%%%%%%%%%%%%%%%%%%%%%%%%%%%%%%%%%%%%%%%%%%%%%%%

\begin{document}
	
	%%%%%%%%%%%%%%%%%%%%%%%%%%%%%%%%%%%%%%%%%%%%%%%%%%%%%%%%%%%%%%%%%%%%%%%%%%%%%%%
	% Nemente:
	%%%%%%%%%%%%%%%%%%%%%%%%%%%%%%%%%%%%%%%%%%%%%%%%%%%%%%%%%%%%%%%%%%%%%%%%%%%%%%%
	\thispagestyle{empty}
	
	{
		\begin{center}
			\sf 
			{\Large Ústav fyziky a technologií plazmatu Přírodovědecké fakulty Masarykovy univerzity} \\
			\bigskip
			{\huge \bfseries FYZIKÁLNÍ PRAKTIKUM} \\
			\bigskip
			{\Large \the\jmenopraktika}
		\end{center}
		
		\bigskip
		
		\sf
		\noindent
		\setlength{\arrayrulewidth}{1pt}
		\begin{tabular*}{\textwidth}{@{\extracolsep{\fill}} l l}
			\large {\bfseries Zpracoval:}  \the\jmeno & \large  {\bfseries Naměřeno:} \the\datum\\[2mm]
			\large  {\bfseries Obor:} \the\obor  \hspace{40mm}  {\bfseries Skupina:} \the\skupina %
			%{\bfseries Ročník:} \the\rocnik \hspace{5mm} {\bfseries Semestr:} \the\semestr  
			&\large {\bfseries Testováno:}\\
			\\
			\hline
		\end{tabular*}
	}
	
	\bigskip
	
	{
		\sf
		\noindent \begin{tabular}{p{3cm} p{0.6\textwidth}}
			\Large  Úloha č. {\bfseries \the\cisloulohy:} \par
			\smallskip
			&\Large \bfseries \the\jmenoulohy  \\[2mm]
		\end{tabular}
	}
	
	\vskip1cm
	
	%%%%%%%%%%%%%%%%%%%%%%%%%%%%%%%%%%%%%%%%%%%%%%%%%%%%%%%%%%%%%%%%%%%%%%%%%%%%%%%
	% konec Nemente.
	%%%%%%%%%%%%%%%%%%%%%%%%%%%%%%%%%%%%%%%%%%%%%%%%%%%%%%%%%%%%%%%%%%%%%%%%%%%%%%%
	
	%%%%%%%%%%%%%%%%%%%%%%%%%%%%%%%%%%%%%%%%%%%%%%%%%%%%%%%%%%%%%%%%%%%%%%%%%%%%%%%
	%%%%%%%%%%%%%%%%%%%%%%%%%%%%%%%%%%%%%%%%%%%%%%%%%%%%%%%%%%%%%%%%%%%%%%%%%%%%%%%
	% Zacatek textu vlastniho protokolu
	%%%%%%%%%%%%%%%%%%%%%%%%%%%%%%%%%%%%%%%%%%%%%%%%%%%%%%%%%%%%%%%%%%%%%%%%%%%%%%%
	%%%%%%%%%%%%%%%%%%%%%%%%%%%%%%%%%%%%%%%%%%%%%%%%%%%%%%%%%%%%%%%%%%%%%%%%%%%%%%%
	
	\begin{multicols}{2}
		\section{Zadání}
		1. Identifikujte spektrální čáry emitované parami materiálu elektrod v obloukovém výboji a určete jejich intenzitu. Ze sklonu pyrometrické přímky určete teplotu oblouku. \\
		2. Určete z naměřeného molekulového spektra radikálu OH rotační teplotu.
		
		
		\section{Teorie}
		
		
		
		Látky excitované na vyšší energetické hladiny mohou svoji energii předat okolí ve formě záření. Pokud se elektron v atomech nebo molekulách přechodem na nižší hladinu deexituje, je vyzářen foton s energií odpovídající rozdílu hladin:
		
		
		\begin{equation}
			h \nu = E_m - E_n = \frac{hc}{\lambda_{mn}}
		\end{equation}
		
		
		Tuto vlastnost využívá optická emisní spektroskopie (OES), která analyzuje záření vzniklé ve vysoce energetických prostředích jako je plazma. Podle struktury spektra lze odlišit typy zářící látky: u atomů vzniká čárové spektrum, u molekul pásové, u pevných látek spojité.
		
		Tato úloha se skládá ze dvou částí. V první určujeme \textbf{excitační teplotu par železa} v obloukovém výboji. Ve druhé vyhodnocujeme \textbf{rotační teplotu radikálu OH} v neizotermickém plazmatu.
		
		\subsection{Excitační teplota atomů železa}
		
		Pro relativní intenzitu spektrální čáry platí vztah:
		
		\begin{equation}			
			I_{mn} \sim \frac{A_{mn} g_m}{\lambda_{mn}} \cdot \exp\left(-\frac{E_m}{kT}\right)		
		\end{equation}
		
		kde $I_{mn}$ je relativní intenzita čáry, $\lambda_{mn}$ vlnová délka, $A_{mn}$ pravděpodobnost přechodu (Einsteinův koeficient spontánní emise), $g_m$ statistická váha horní hladiny, $E_m$ excitační energie horní hladiny a $k$ je Boltzmannova konstanta, $T$ teplota.
		
		
		Po úpravách získáme vztah vhodný pro lineární závislost tzv. pyrometrické přímky:
		
		
		\begin{equation}
			\ln\left(\frac{I_{mn} \cdot \lambda_{mn}}{A_{mn} g_m}\right) = -\frac{E_m}{kT} + \text{konst}
		\end{equation}
		
		
		Z naměřených relativních intenzit, známých vlnových délek, energií a přechodových pravděpodobností lze tak sestavit lineární závislost a určit teplotu z její směrnice.
		
		\subsection{Rotační teplota molekuly OH}
		
		Dvouatomová molekula má oproti atomům navíc vibrační a rotační stupně volnosti. Celková energie jejího stavu je dána jako součet:
		
		
		\begin{equation}
			E = E_e + E_v(\nu) + E_r(N)
		\end{equation}
		
		
		kde $E_e$ je energie elektronového stavu, $E_v(\nu)$ vibrační energie (v aproximaci anharmonického oscilátoru) a $E_r(N)$ rotační energie (v aproximaci netuhého rotátoru). Pro rotační energii platí:
		
		
		\begin{equation}
			E_r(N) = hc \cdot \left[B_\nu N(N+1) - D_e N^2(N+1)^2\right]
		\end{equation}
		
		
		Základ pro výpočet rotační teploty tvoří intenzity rotačních čar:
		
		
		\begin{equation}
			I_{J'} \propto \tilde{\nu}^4 S_{J'J''} \cdot \exp\left(-\frac{B' hc N'(N'+1)}{kT}\right)
		\end{equation}
		
		
		Po logaritmování opět dostáváme lineární vztah, jehož směrnice umožňuje výpočet teploty:
		
		
		\begin{equation}
			\ln\left(\frac{I_{J'}}{\tilde{\nu}^4 S_{J'J''}}\right) = -\frac{B' hc}{kT} N'(N'+1) + \text{konst}
		\end{equation}
		
		
		kde $\tilde{\nu}$ je vlnočet čáry, $S_{J'J''}$ je Hoenl-Londonův faktor, $B'$ rotační konstanta horního vibračního stavu a $N'$ je rotační kvantové číslo (v této úloze platí $N' = J' - 1/2$).
		
		
		\subsection{Vyhodnocení a software}
		
		Při praktickém vyhodnocení se provádí je nutná správná identifikace spektrálních čar (včetně případného posunu) a odečtení temného signálu. Zatímco temný signál lze eliminovat bez referenčních dat, posun spektrálních čar je nutno vyvodit za pomocí reference.

		
		Pro identifikaci rotační struktury spektra OH se využívá programů \textbf{SPAN 1.7} a \textbf{Lifbase 2.1} – první umožňuje analýzu naměřeného spektra, druhý poskytuje simulované spektrum pro orientaci. Hodnota rotační konstanty použitá v úloze je:
		
		\[
		B' = 1696.6\ \text{cm}^{-1}
		\]
		
		
		\section{Zpracování dat}
		\subsection{Excitační teplota atomů železa}
		Zpracováváme soubor \verb*|data3|, pracujeme se spektrem viditelným na obr. 1.
		\begin{figure}[H]
			\centering
			\includegraphics[width=0.9\linewidth]{spectrum}
			\caption{Spektrum obloukového výboje železa - závislost rel. intenzity na vlnové délce}

		\end{figure}
		Pro kompenzaci temného signálu využíváme oblasti bez spektrálních čar - zde konkrétně plochou oblast kolem vlnové délky 450 nm. Detekcí vrcholů nalezneme potenciální spektrální čáry, poté je přiřadíme k tabulce v návodu\cite{navod}. Získáváme data na obr. 2:
		
		\begin{figure}[H]
			\centering
			\includegraphics[width=0.9\linewidth]{spectrum2}
			\caption{Spektrum obloukového výboje železa - po odstranění temného signálu a detekci špiček}

		\end{figure}
		
		Následně přiřazujeme detekované čáry k referenčním čarám v návodu, pro získání rel. intenzity integrujeme oblast pod špičkou. Pro integraci je nutno vymezit meze samotné čáry - jelikož se proces snažíme algoritmizovat a data nezpracovávat manuálně, meze definujeme tak, že pro každou čáru detekujeme i Full Width Half Maxmum a samotné meze jako body vzdálené od středu o $1.625\cdot \mathrm{FWHM}$ - tuto hodnotu volíme, jelikož jsou při ní správně vymezeny veškeré špičky bez složitější analýzy. Zpracováním získáváme výsledek viditelný na obr. 3.
		\begin{figure}[H]
			\centering
			\includegraphics[width=0.9\linewidth]{spectrum3}
			\caption{Spektrum obloukového výboje železa - fialové oblasti znázorňují meze integrace}			
		\end{figure}
		Zároveň lze měřit odchylku detekovaných čar od čar referenčních, získáváme \\ $\Delta \lambda = 0.084\pm 0.151 \,\mathrm{nm}$. Odchylka od referenčních dat tedy není primárně způsobena posunem celého spektra. Ze závislosti odchylky na vlnové délce pozorujeme, že s rostoucí vlnovou délkou roste i odchylka (obr. 4). Lze tedy předpokládat, že jsou data "roztažena", odchylka roste od bodu, který jsme lineární regresí získali jako průsečík osy x.
		\begin{figure}[H]
			\centering
			\includegraphics[width=0.9\linewidth]{zavislost}
			\caption{Závislost odchylky měřené vlnové délky od délky referenční}			
		\end{figure}
		Se získanými relativními intenzitami je nicméně možno určit teplotu výboje. Použitím formule (3), resp. vykreslením pyrometrické přímky (obr. 5), odhadujeme teplotu.
		\begin{figure}[H]
			\centering
			\includegraphics[width=0.9\linewidth]{pyrometrika}
			\caption{Pyrometrická přímka obloukového výboje železa}			
		\end{figure}
		\begin{equation*}
			T_{\mathrm{Fe}} = 5729 \pm 564 \,\mathrm{K}
		\end{equation*}
		Při zpracování v programu SPAN získáváme teplotu:
		\begin{equation*}
			T_{\mathrm{Fe}} = 5624 \pm 965 \,\mathrm{K}
		\end{equation*}
		Zabudovaná funkce kalkulace teploty nám nefungovala, proto jsme tuto teplotu získali znovu kalkulací intenzit a následnou tvorbou pyrometrické přímky. Vysoká nejistota vychází mj. ze skutečnosti, že jsme tímto způsobem získali intenzitu relativně nízkého počtu čar.
		\subsection{Rotační teplota molekuly OH}
		Pro analýzu spektra molekuly OH nejdříve generujeme referenční data tak, jak je uvedeno v návodu. Okamžitě pozorujeme velký rozdíl mezi měřenými a referenčními (simulovanými) daty, jak je vidět na Obr. 6, odchylka zde dosahuje až několika nanometrů.
		\begin{figure}[H]
			\centering
			\includegraphics[width=0.9\linewidth]{srovnani}
			\caption{Naměřené a simulované spektrum OH}			
		\end{figure}
		Kalibraci provádíme obdobným způsobem, jako v předchozí úloze. Srovnáním největších špiček určíme odchylku, lineárním fitem závislosti odchylky na vlnové délce (původní, nekalibrované) z konstant regrese určíme kalibrační funkci, kterou aplikujeme na původní vlnové délky naměřených dat. Zároveň odstraňujeme hodnoty temného signálu (v tomto případě z oblasti vlnových délek kolem 300 nm), stejným způsobem jako v předchozí úloze. Srovnání kalibrovaného a simulovaného spektra je tímto úspěšně provedeno, jak lze ověřit na Obr. 7.
		\begin{figure}[H]
			\centering
			\includegraphics[width=0.9\linewidth]{Spectrum4}
			\caption{Kalibrované a simulované spektrum OH}			
		\end{figure}
		Nyní referenční data z tabulky návodu přiřadíme konkrétním čarám, integrací získáme jejich rel. intenzitu a znovu můžeme určit rotační teplotu pyrometrickou přímkou (obr. 8)
		\begin{figure}[H]
			\centering
			\includegraphics[width=0.9\linewidth]{pyrometrika2}
			\caption{Pyrometrická přímka OH radikálu}			
		\end{figure}
		\begin{equation*}
			T_{\mathrm{OH}} = 280 \pm 1 \,\mathrm{K}
		\end{equation*}
		\subsubsection{Zpracování dat v programu SPAN}
		
		Pro výpočet rotační teploty molekuly OH data znovu importujeme, v programu dle instrukcí označíme pouze radikál OH, následně identifikujeme konkrétní čáry. Následně lze automaticky ze spektra kalibrovat graf, teplota je vypočtena automaticky. Získáváme:
		\begin{equation*}
			T_{\mathrm{OH}}= 300 \pm 16 \,\mathrm{K}
		\end{equation*}
		
		
		
		
		\section{Závěr}
		Úspěšně se nám podařilo stanovit teplotu elektrického výboje Fe - získaná hodnota pro teplotu výboje dává smysl - teplota elektrického oblouku dosahuje hodnot 3\,000 -- 8\,000 K \cite{vut}. Stejně tak se nám podařilo určit rotační teplotu molekuly OH, jejíž hodnotu se nám zároveň podařilo ověřit, respektive replikovat, softwarem SPAN.
		
\printbibliography
\newpage

			\end{multicols}
			\section{Příloha}
			\begin{table}[H]
				\centering
				\caption{Naměřené a referenční hodnoty spektrálních čar železa včetně intenzit a dalších parametrů.}
				\label{tab:zelezo_spektrum}
				\begin{tabular}{lrlllrl}
					$\lambda_\mathrm{naměř.}$ [nm] &  $I_\mathrm{max}$ [a.u.] &  $\lambda_\mathrm{ref}$ [nm] &  $E_m$ [eV] &  $A_m g_m$ [$10^8\,\mathrm{s}^{-1}$] &  $I_\mathrm{int}$ [a.u.] &  $\Delta \lambda$ [nm] \\  \hline                    429.5289 &              5451.388889 &                      429.413 &       4.371 &                               0.7100 &              1203.107744 &                -0.1159 \\                      430.0347 &              8610.388889 &                      429.924 &       5.308 &                               5.2000 &              1931.572144 &                -0.1107 \\                 430.8777 &             15041.388889 &                      430.791 &       4.434 &                               5.9000 &              3821.806067 &                -0.0867 \\                       431.5521 &              3606.388889 &                      431.509 &       5.070 &                               1.5000 &               765.422144 &                -0.0431 \\                       432.6198 &             15763.388889 &                      432.576 &       4.473 &                               6.1000 &              3630.195206 &                -0.0438 \\                       433.6876 &              2578.388889 &                      433.705 &       4.415 &                               0.2300 &               584.124067 &                 0.0174 \\                       435.2611 &              2262.388889 &                      435.274 &       5.070 &                               1.0000 &               511.688678 &                 0.0129 \\                       437.5088 &              1956.388889 &                      437.593 &       2.832 &                               0.0094 &               466.247128 &                 0.0842 \\                       438.2954 &             20480.388889 &                      438.357 &       4.312 &                               7.7000 &              5367.759939 &                 0.0616 \\                       440.3742 &             13880.388889 &                      440.475 &       4.371 &                               4.4000 &              3722.427900 &                 0.1008 \\                       441.3855 &              8525.388889 &                      441.512 &       4.415 &                               2.8000 &              1992.000750 &                 0.1265 \\                        442.5652 &              2317.388889 &                      442.731 &       2.851 &                               0.0099 &               538.930667 &                 0.1658 \\                        444.0256 &              3902.388889 &                      444.234 &       4.988 &                               1.1000 &              1030.196411 &                 0.2084 \\                        444.5311 &              2309.388889 &                      444.772 &       5.009 &                               1.1000 &               669.295872 &                 0.2409 \\                        445.7104 &              3126.388889 &                      445.912 &       4.955 &                               1.0000 &              1532.649372 &                 0.2016 \\                        446.4405 &              3595.388889 &                      446.655 &       5.606 &                               5.3000 &              1298.992717 &                 0.2145 \\                        447.3389 &              3947.388889 &                      447.602 &       5.614 &                               5.4000 &               852.832389 &                 0.2631 \\                        447.9565 &              3371.388889 &                      448.217 &       2.875 &                               0.0053 &              1016.836294 &                 0.2605 \\                      449.1917 &              3767.388889 &                      449.457 &       4.955 &                               1.2200 &               844.602400 &                 0.2653 \\                        452.5592 &              6224.388889 &                      452.862 &       4.913 &                               1.8000 &              1978.389039 &                 0.3028 \\
				\end{tabular}
			\end{table}
			
			\begin{table}[H]
				\centering
				\caption{Naměřené spektrální čáry radikálu OH a hodnoty potřebné pro určení rotační teploty.}
				\label{tab:oh_spektrum}
				\begin{tabular}{lrllllll}

					$\lambda_\mathrm{naměř.}$ [nm] &
					$I_\mathrm{max}$ [a.u.] &
					$\lambda_\mathrm{ref}$ [nm] &
					$N'$ &
					$S_{J'J''}$ &
					$I_\mathrm{int}$ [a.u.] &
					$N'(N'+1)$ &
					$\ln \left(\frac{I_\mathrm{int}}{\tilde\nu^4 S_{J'J''}}\right)$ \\
					\hline

					307.846 & 38365.416 & 307.843 & 1.0 & 0.563 & 699.391 & 2.0 & -34.429 \\
					307.997 & 40643.416 & 307.996 & 2.0 & 1.065 & 803.388 & 6.0 & -34.926 \\
					308.329 & 21662.749 & 308.326 & 4.0 & 2.100 & 425.078 & 20.0 & -36.237 \\
					308.521 & 11930.749 & 308.512 & 5.0 & 2.640 & 235.717 & 30.0 & -37.053 \\
					308.736 &  5361.082 & 308.733 & 6.0 & 3.160 & 112.294 & 42.0 & -37.972 \\

				\end{tabular}
			\end{table}
			
		
		
		% Nakonec nezapomeňte projet text programem vlna nebo vlnka, např.
		% 	vlna -m -l -n mojeuloha.tex
		% nebo zkontrolovat a opravit jednopísmenné předložky na koncích řádků ručně.

\end{document}
